% Lukas Gratwohl
% 
% Testen. Kochen. Geniessen.
%
% Gesammelte Rezepte

\documentclass[a4paper,12pt,titlepage]{article}

\usepackage[utf8]{inputenc}		%Darstellung Umlaute
\usepackage[german]{babel}		%Deutsche Trennregeln usw.
\usepackage{array}			%Tabellen formatieren
\usepackage{tabularx}
\usepackage[top=4cm, bottom=3cm, left=3.5cm, right=2.5cm]{geometry}
					%Ränder festlegen
\usepackage{amssymb}			%Für Durchmesserzeichen
\usepackage{fancybox}
\usepackage{fancyhdr}
\usepackage{graphicx}

%Ränder um Bilder
%\usepackage{float}
%\floatstyle{boxed} 
%\restylefloat{figure}

\usepackage{multirow}
\usepackage{hyperref}

\newcommand{\shorttopPic}[6]{
\vspace*{0.3cm}
\fbox{
	\begin{tabularx}{0.965\linewidth}{Xll}
		\addtolength{\linewidth}{0.35cm} \multirow{3}{*}{\hspace*{-0.35cm}\includegraphics[width = \linewidth, trim = 0cm #3 0cm #2, clip = true]{#1}} & Zubereitung & #4\\
		& Menge & #5\\
		& Quelle & #6
	\end{tabularx}
}
\vspace*{0.4cm}
\newline
}


\newcommand{\shorttop}[3]{
\vspace*{0.3cm}
\begin{flushright}
\fbox{
\begin{tabular}{ll}
		Zubereitung & #1\\
		Menge & #2\\
		Quelle & #3
	\end{tabular}
}
\end{flushright}
\vspace*{0.4cm}
}

\newcommand{\selfmadetitlepage}{
\pagestyle{empty}
\begin{center}
\vspace*{6cm}
\begin{Huge}
\textbf{Erleben. Kochen. Geniessen.}
\end{Huge}\\
\vspace*{0.5cm}
---------------------\\

\vspace*{1cm}
\begin{Large}
\textbf{Gesammelte Rezepte}
\end{Large}

\vspace*{2.5cm}
\begin{Large}
Lukas Gratwohl
\end{Large}\\

\vspace*{1cm}
\today
\end{center}
\newpage
}



%\newcommand{\shorttop}[3]{
%\vspace*{0.5cm}
%\begin{flushleft}
%	\fbox{\parbox{0.5\linewidth}{
%			\begin{tabular}{ll}
%			Zubereitung & #1\\
%			Menge & #2\\
%			Quelle & #3
%		\end{tabular}
%		}
%	}
%\end{flushleft}
%\vspace*{0.7cm}
%}

% FANCYHEADER EINSTELLUNGEN
\newcommand{\setfancyVorwort}{
  \fancyhead{}
  \fancyfoot{}
  \fancyfoot[C]{{-~\thepage~-}}
  \renewcommand{\headrulewidth}{0pt}
  \renewcommand{\footrulewidth}{0pt}
}

\newcommand{\setfancy}{
  \fancyhead{}
  \fancyfoot{}
  \fancyhead[L]{{\small\scshape Gesammelte Rezepte}}
  \fancyhead[C]{{\small\scshape Lukas Gratwohl}}
  \fancyhead[R]{\small\scshape Apero}
  \fancyfoot[C]{{-~\thepage~-}}
  \renewcommand{\headrulewidth}{0.1pt}
  \renewcommand{\footrulewidth}{0pt}
}




% MULTICOLS EINSTELLUNGEN
%\setlength{\columnsep}{6mm}
%\setlength{\columnseprule}{.1pt}
% Column Separation Rules
%\newcommand{\csr}{\setlength{\columnseprule}{.1pt}}
%\newcommand{\nocsr}{\setlength{\columnseprule}{0pt}}
% Spaltenumbruch
%\newcommand{\brk}{\columnbreak}

% TEXTHÖHE
%\setlength{\textheight}{245mm}

% ABSTAND ZUR FUSSZEILE
%\setlength{\footskip}{10mm}
%\setlength{\headsep}{10mm}

% KEIN EINZUG ZU BEGINN EINES NEUEN PARAGRAPHEN
\setlength{\parindent}{0mm}

% DEHNUNG ALLER ARRAYS (bessere Lesbarkeit)
\renewcommand{\arraystretch}{1.2}
% TIEFE DES INHALTSVERZEICHNISSES
\setcounter{tocdepth}{3}

% ABSTAND NACH TITELN
%\titlespacing*{\subsection}{0pt}{-5pt}{-10pt}[0pt]
%\titlespacing*{\subsubsection}{0pt}{-5pt}{-10pt}[0pt]
%\titlespacing*{\section}{0pt}{-5pt}{-5pt}[0pt]

% AUFZÄHLUNGEN
%\setlength{\itemsep}{-2mm}
%\setlength{\topsep}{-2mm}
%\setlength{\partopsep}{-2mm}




\title{\huge Entdecken. Kochen. Geniessen.\\\lage Gesammelte Rezepte}
\author{Lukas Gratwohl}
\date{\today}


\begin{document}

%Titelseite
\selfmadetitlepage

%Inhaltsverzeichnis
\pagestyle{empty}
\tableofcontents

%Erste Seite
\newpage
\setcounter{page}{1}
\pagestyle{fancy}
\setfancyVorwort

\textbf{\LARGE Vorwort}
\\
\\
Das kleine aber feine Rezeptbuch in deinen Händen entstand über Jahre und beinhaltet Rezepte, welche mich geprägt und berührt haben. Genauso wie die Menschen, von denen sie stammen.\\\\

\newpage
\setfancy

%
%
%
%Vorspeisen
%
%
%
\part{Ap\'{e}ro}

%Guacamole
\section{Guacamole}
\shorttopPic{Bilder/guacamole}{3.45cm}{4.13cm}{10 Minuten}{2 Personen}{Andrea Frei}
Ob zu Taquitos, Tortilla-Chips oder als Beilage zu Fleisch: Mit diesem herrlichen Dip aus der mexikanischen Küche wird jedes Mahl in eine leidenschaftliche Latin-Kreation verwandelt.
\\
\\
\\
\begin{tabularx}{\textwidth}{r>{\bfseries\textbf}lX}
	1 & Avocado & halbieren und mit einem Esslöffel aushöhlen. Mit einer Gabel pürieren.\\
	1 & Tomate & und\\
	${}^1/_2$ & Zwiebel & in kleine Würfel schneiden, \\
	${}^1/_2$ & Limette & auspressen und mit\\
	1 Prise & Salz & gut mit der Avocadomasse vermischen.

\end{tabularx}
\\
\\
\\
\textbf{Tipps}
\begin{itemize}
	\item Mit Nachos von Coop Fine Food besonders gut.
	\item Für eine besondere Note kann je nach Gusto auch Peterli hinzugefügt werden.
\end{itemize}

\newpage


%
%
%
%Hauptspeisen
%
%
%
\fancyhead[R]{\small\scshape Hauptspeisen}

\part{Hauptspeisen}

%Knöpfli
\section{Knöpfli}
\shorttopPic{Bilder/knoepfli}{3cm}{13.45cm}{10 Minuten}{4 Personen}{Grossmami}
Ein wunderbares Rezept für die leichte Küche. Anstelle von mastigem Mehl wird hier Quark verwendet. Einfach und genial.
\\
\\
\\
\begin{tabularx}{\linewidth}{r>{\bfseries\textbf}lX}
	& & 1.5 l Wasser aufsetzen.\\
	400 g & Knöpflimehl\\
	400 g & Quark\\
	6 & Eier & Alles mischen bis Masse gleichmässig körnig aussieht. Kochendes Wasser salzen und Masse mit einem Handschaber durch ein Knöpflisieb streichen.\newline \newline
	Etwa zwei Minuten kochen lassen und mit einem Sieb herausfischen.
\end{tabularx}
\\
\\
\\
\textbf{Tipps}
\begin{itemize}
	\item Bei dieser Menge Knöpfli müssen eventuell zwei Kochdurchgänge eingeplant werden.
	\item Die figurbewussten Leserinnen und Leser können selbstverständlich auch Magerquark verwenden.
\end{itemize}

\newpage

%Pasta fatta in casa
\section{Pasta fatta in casa}
%\shorttopPic{Bilder/pastaFattaInCasa}{4.5cm}{6.55cm}{10 + 120 Minuten}{3 Personen}{Grossmami}
%\shorttopPic{Bilder/pastaFattaInCasa}{6.5cm}{4.55cm}{10 + 120 Minuten}{3 Personen}{Grossmami}
%\shorttopPic{Bilder/pastaFattaInCasa3}{5.5cm}{15.8cm}{10 + 120 Minuten}{3 Personen}{Grossmami}
\shorttopPic{Bilder/pappardelle}{3cm}{6.4cm}{10 + 120 Minuten}{3 Personen}{Grossmami}
Was wäre die Italienische Küche ohne die hausgemachte Pasta? Durch das Olivenöl in diesem Rezept wird die Pasta wunderbar geschmeidig und kann für alle erdenklichen Pastaversionen verwendet werden.
\\
\\
\\
\begin{tabularx}{\linewidth}{r>{\bfseries\textbf}lX}
	400 g & Weissmehl\\
	4 & Eier\\
	4 Esslöffel & Olivenöl\\
	5 g & Salz & Alles gut durchkneten bis Teig geschmeidig wird. Teig zu einer Kugel formen und in Klarsichtfolie mindestens zwei Stunden im Kühlschrank aushärten lassen.\newline \newline
	Teig auswalen und beliebig formen. Hier sind der Fantasie keine Grenzen gesetzt.
\end{tabularx}
\\
\\
\\
\textbf{Tipps}
\begin{itemize}
	\item Am einfachsten und schnellsten zu machen sind die Pappardelle (breite Bandnudeln). Sehen äusserst gut aus und erzeugen auf dem Teller diesen gewissen Selbstgemacht-Effekt.
\end{itemize}
\newpage

%Pasta alla Norma
\section{Pasta alla Norma}
\shorttopPic{Bilder/pastaAllaNorma}{8.5cm}{19cm}{30 Minuten}{4 Personen}{Stügge}
Auf Sizilien ist Pasta alla Norma ein Klassiker. Wirklich jeder kennt dieses Gericht. Das perfekte Mahl zu einem Nero d'Avola und der gleichnamigen Oper von Vincenzo Bellini.
\\
\\
\\
\begin{tabularx}{\linewidth}{r>{\bfseries\textbf}lX}
	2 & Auberginen & längs vierteln, Samen entfernen und in fingergrosse Streifen schneiden.\newline Mit qualitativ gutem\\
	& Olivenöl & und\\
	1 Esslöffel & Oregano & anbraten bis sie rundum goldbraun sind. Auf mittlere Hitze schalten.\\
	4 Zehen & Knoblauch & in feine Scheiben schneiden und beigeben.\\
	1 Bund & Basilikum & folgendermassen auftrennen: Stängel hacken und beigeben, Blätter auf die Seite legen.\\
	800 g & Pelati & zerkleinern und mit einem Schuss Aceto Balsamico hinzufügen. Die Sauce 10-15 Minuten köcheln lassen.\newline Die Hälfe der Basilikumblätter zerpflücken und zusammen mit Salz und Pfeffer untermischen.\\
	500 g & Spaghetti & gemäss Anleitung al dente kochen. Einen kleinen Teil des Kochwassers vor dem Abgiessen zur Sauce hinzufügen.\newline 
	Die Spaghetti mit der Hälfte der Sauce mischen und auf den Tellern zu Nestern drehen. Die andere Hälfte der Sauce in die Nester füllen und mit\\
	150 g & Ricotta & und dem übrigen Basilikum verzieren.
\end{tabularx}
\\
\\
\\
\textbf{Tipps}
\begin{itemize}
	\item Bei zu vielen Auberginen müssen eventuell mehrere Anbratdurchgänge eingeplant werden.
	\item Als Alternative zum frischen Ricotta kann auch Parmesan, Pecorino oder gesalzener Ricotta verwendet werden.
\end{itemize}
\newpage

%Hähnchen Tikka Masala
%\section{Hähnchen Tikka Masala}
%\begin{tabular}{ll}
%	Zubereitung: & ... Minuten\\
%	Menge: & ... Person\\
%	Quelle: & Sushila
%\end{tabular}

%\newpage

%Melanzane alla parmigiana
\section{Melanzane alla parmigiana}
\shorttopPic{Bilder/melanzane}{4cm}{3.25cm}{40 + 30 Minuten}{6 Personen}{Stügge}
Ob als pompöser Hauptgang oder als elegante Beilage - Auberginen, Tomaten und Parmesan ergeben einen Auflauf, von dem man einfach nicht genug bekommen kann.
\\
\\
\\
\begin{tabularx}{\linewidth}{r>{\bfseries\textbf}lX}
	3 & Auberginen & in 1 cm dicke Scheiben schneiden und beiseite stellen.\\
	& Olivenöl & in einem grossen Topf erhitzen.\\
	1 & Zwiebel & und\\
	1 Zehe & Knoblauch & feinhacken und bei mittlerer Temperatur mit\\
	1 Teelöffel & Oregano & 10 Minuten dünsten. Der Knoblauch sollte jetzt leicht gebräunt sein.\\
	800 g & Pelati & zerkleinern und hinzufügen. Deckel auflegen und die Sauce 10-15 Minuten köcheln lassen. Den Backofen auf 190 $^{\circ}$C vorheizen.\newline \newline Die Hälfte der Basilikumblätter zerpflücken und zusammen mit Salz und Pfeffer untermischen. Wenn die Tomatensauce zu einem süsslich duftenden Konzentrat eingekocht ist, mit Salz und Peffer abschmecken.\newline \newline Die Auberginen von beiden Seiten anbraten.\\
	1 Hand & Basilikum & Blätter abzupfen und mit einem Spritzer Aceto Balsamico zur Sauce hinzufügen.\newline \newline In einer Gratinform etwas Tomatensauce verteilen.\\
	250 g & Parmesan & frisch gerieben. Einen Teil über die Tomatensauce streuen. Jetzt eine Lage gegrillte Auberginen darüber legen. In dieser Reihenfolge die Zutaten einfüllen. Eine dünne Schicht Sauce und eine ordentliche Lage Parmesan bilden den Abschluss.\newline \newline Die Melanzane für 30 Minuten im Ofen goldbraun backen lassen bis die Sauce träge blubbert.\\
\end{tabularx}
\\
\\
\\
\\
\\
\textbf{Tipps}
\begin{itemize}
	\item Die Auberginen grillen, wodurch sie innen beinahe cremig werden, während sie beim Braten in Öl ziemlich viel Fett aufsaugen.
	\item Nach Belieben kann die Sauce nach dem Basilikum auch püriert werden.
	\item Auf die oberste Lage eignen sich zuätzlich Büffelmozarella und/oder Brotkrumen.
\end{itemize}
\newpage

%Nudelauflauf
%\section{Nudelauflauf}
%\begin{tabular}{ll}
%	Zubereitung: & ... Minuten\\
%	Menge: & ... Person\\
%	Quelle: & Lenny
%\end{tabular}
%\newpage

%Pizzateig
\section{Pizzateig}
\shorttopPic{Bilder/pizzateig}{3cm}{3.72cm}{10 + 30 Minuten}{1 Blech}{trial and error}
Was wäre eine Pizza ohne einen selbstgemachten Pizzateig? Es lohnt sich, diese zusätzlichen 10 Minuten in Kauf zu nehmen.
\\
\\
\\
\begin{tabularx}{\linewidth}{r>{\bfseries\textbf}lX}
	250 ml & Wasser & lauwarm bis heiss.\\
	15 ml & Olivenöl & \\
	42 g & Hefe & \\
	10 g & Salz & \\
	1 Prise & Zucker & Alles gut mischen, bis sich die Hefe ganz aufgelöst hat.\\
	500 g & Weissmehl & in Portionen darunterkneten, bis sich der Teig nicht mehr klebrig anfühlt. In einer grossen Schüssel mit einem Tuch abgedeckt 30 Minuten an einem warmen Ort ruhen lassen.
\end{tabularx}
\\
\\
\\
\textbf{Tipps}
\begin{itemize}
	\item Beim Zucker gehen die Meinungen weit auseinander. Ich habe beides ausprobiert und habe mich für die Variante mit Zucker entschieden. Der Teig wird aromatischer und dank der Karamelisierung ein wenig knuspriger.
	\item Die Mengenangaben stimmen ziemlich gut. Vielfach arbeite ich gleich mit den ganzen 500\,g Mehl.
\end{itemize}

\newpage

%Peperoni al forno
\section{Peperoni al forno}
\shorttopPic{Bilder/peperoni}{4.5cm}{7.55cm}{10 + 35 Minuten}{1 Person}{Mam}
Wenn die Peperoni wie hier beschrieben zubereitet werden, bekommen sie eine leicht süssliche Note und überraschen mit einem wunderbar aromatischen Geschmack.
\\
\\
\\
\begin{tabularx}{\linewidth}{r>{\bfseries\textbf}lX}
	& & Ofen auf 180 $^{\circ}$C Ober-/Unterhitze vorheizen.\\
	3 grosse & Peperoni & in 1 cm breite Streifen schneiden.\newline Ein Blech mit Backpapier auslegen und die Peperonistreifen gleichmässig verteilen.\\
	& Olivenöl & in dünnen Fäden darübergiessen. Es genügen drei Querbewegungen.\\
	& Salz & und\\
	frischen & Pfeffer & grosszügig über das ganze Blech streuen.\newline \\
	& & Peperoni 30 Minuten in der Mitte des vorgeheizten Ofen backen lassen.\\
\end{tabularx}
\\
\\
\\
\textbf{Tipps}
\begin{itemize}
	\item Besonders gut machen sich diese Peperoni al forno als Beilage zu einem Risotto mit viel Parmigiano.
	\item Falls die Streifen ein wenig breiter geworden sind, muss die Backzeit um etwa 10 Minuten erhöht werden.
\end{itemize}
\newpage


%
%
%
%Desserts
%
%
%
\fancyhead[R]{\small\scshape Desserts}

\part{Desserts}

%Kaiserschmarrn
\section{Kaiserschmarrn}
\shorttopPic{Bilder/kaiserschmarrn}{11cm}{10.1cm}{15 Minuten}{1 Person}{Mimi Feichtner}
Hier ein Rezept für alle Fans der Österreichischen Küche. Ein Klassiker, der es versteht, immer wieder aufs Neue zu begeistern.
\\
\\
\\
\begin{tabularx}{\linewidth}{r>{\bfseries\textbf}lX}
	100 g & Weissmehl\\
	1 Prise & Salz\\
	2 & Eier\\
	1 dl & Vollmilch\\
	1 Gutsch & Rahm & Alles in eine Schüssel geben und vorsichtig umrühren. Kleine Klümpchen werden hier toleriert.\\
	20 g & Butter & in Bratpfanne heiss werden lassen und Masse beigeben. Anziehen lassen und wenden. Wieder anziehen lassen und vorsichtig mit einer Kehle in Schnipsel zerschneiden.\newline \newline
	Glodbraun anbraten und mit etwas Puderzucker servieren.
\end{tabularx}
\\
\\
\\
\textbf{Tipps}
\begin{itemize}
	\item Beim Wenden Kaiserschmarrn in Viertel teilen. Vereinfacht das Wenden erheblich.
	\item Als Beilage eignen sich Beeren in allen Formen. Im Tirol wird der Kaiserschmarrn mit Preiselbeeren oder Zwetschgenröster (Zwetschgenkompott) serviert.
\end{itemize}
\newpage

%Granita di mandorle
\section{Granita di mandorle}
\shorttopPic{Bilder/granitaDiMandorle}{14cm}{12.4cm}{10 + 300 Minuten}{1.5 Liter}{trial and error}
Granita ist eine gefrorene sizilianische Süssspeise mit einer granulatähnlichen Konsistenz. Die gängigen Varianten werden mit Zitrone oder Mandeln gemacht. Wunderbar an einem heissen Morgen zusammen mit einem Briosch und einem Ristretto.
\\
\\
\\
\begin{tabularx}{\linewidth}{r>{\bfseries\textbf}lX}
	1.5 Liter & Wasser & in einem Kochtopf auf niedriger Stufe warm werden lassen.\\
	200 g & Mandelmarzipan & und\\
	150 g & Zucker & langsam rührend darin auflösen.\newline \newline
		Das entstandene Zuckerwasser in eine Metallschüssel geben und ins Gefrierfach stellen.\newline \newline
		Mit einer Periode von 60 Minuten die Masse mit einem Schwingbesen umrühren. Nach etwa fünf Stunden erreicht die Granita die gewünschte Konsistenz.	
\end{tabularx}
\\
\\
\\
\textbf{Tipps}
\begin{itemize}
	\item Der Mandelmarzipan sollte mindestens 50\% Mandeln enthalten.
	\item Die Granita nicht zu lange aus dem Gefrierfach nehmen. Sonst bildet sich ein einziger riesiger Mandelwasserklumpen.
\end{itemize}
\newpage


%Baileys Mousse
\section{Baileys Mousse}
%\shorttopPic{Bilder/baileysMousse}{20cm}{6.6cm}{20 Minuten}{4 Personen}{QimiQ Rezeptbuch}
\shorttopPic{Bilder/baileysMousse2}{0.7cm}{2.24cm}{20 Minuten}{4 Personen}{QimiQ Rezeptbuch}
Ein Abgang par Excellence. Eine süsse Verführung, die auf der Zunge vergeht und lange in Erinnerung bleibt. Die hier vorgestellte Art ist für Geniesser ein wahres Vergnügen und lässt sich in zahlreichen Versionen neu erleben.
\\
\\
\\
\begin{tabularx}{\linewidth}{r>{\bfseries\textbf}lX}
	250 g & QimiQ & zimmerwarm, glatt rühren.\\
	5 dl & Milch & beigeben.\\
	6 Esslöffel & Baileys & mit\\
	2 Esslöffel & Zucker & dazugeben und alles gut mischen.\\
	1 dl & Vollrahm & steif schlagen und darunter ziehen.\newline \newline
	In Form giessen, mit Folie abdecken und im Kühlschrank steif werden lassen.
\end{tabularx}
\\
\\
\\
\textbf{Tipps}
\begin{itemize}
	\item Gläser mit saugfähigen Biskuits auslegen, Masse darüber giessen und einzeln mit Folie abdecken. In Einzelportionen erspart man sich das unschöne Schöpfen und kann es gemütlich aus dem eigenen Glas geniessen.
\end{itemize}
\newpage

%Brätzeli
\section{Brätzeli}
\shorttopPic{Bilder/braetzeli}{1.6cm}{10cm}{20 + 180 + 40 Minuten}{1 Schachtel}{Lydia Thilmann}
Das Phantastische an den Brätzeli ist ihre Flexibilität. Ob zu Weihnachten, Ostern oder als kleines Geburtstagsgeschenk. Dieses Rezept bereitet das ganze Jahr hindurch Freude.
\\
\\
\\
\begin{tabularx}{\linewidth}{r>{\bfseries\textbf}lX}
	250 g & Butter & zimmerwarm, geschmeidig rühren.\\
	4 & Eier & und\\
	600 g & Mehl & beigeben und durchrühren. Von\\
	2 & Zitronen & die Schale abreiben und hinzufügen.\\
	& & Teig durchkneten bis eine gleichmässig helle Masse entsteht. Den fertigen Teig mehrere Stunden kühl stellen.\newline \newline
		Das Brätzeli-Eisen vorheizen und währenddessen die Masse in 2 cm dicke Kugeln formen. Wenn das Eisen heiss ist, jeweils eine Kugel auf einen Quadranten legen und das Brätzeli-Eisen fest zudrücken. Je nach Brätzeli-Eisen etwa 15 s backen lassen. Hier ist gutes Zeitgefühl angesagt. Auch wird das Eisen mit der Zeit immer heisser, was die Backzeit verkürzt. Mit einem Spachtel die Brätzeli zum Auskühlen auf ein Gitter legen.
\end{tabularx}
\\
\\
\\
\textbf{Tipps}
\begin{itemize}
	\item Die Brätzeli bleiben in einer luftdichten Alubox am längsten halt- und geniessbar.
	\item Statt die Brätzeli flach auskühlen zu lassen, können sie direkt nach dem Backen auf eine Holzkehle gerollt werden. Nach dem Auskühlen können sie so mit allerlei feinen Moussen oder Crèmen gefüllt werden.
	\item Frei nach Freiburger Art kann auch noch Weisswein beigemischt werden. Dann wird die Masse jedoch flüssig und kann nicht mehr zu Kugeln geformt werden.
\end{itemize}
\newpage

%Sablé
\section{Sablé}
\shorttopPic{Bilder/sable}{14cm}{10.45cm}{60 + 120 + 25 Minuten}{4 Personen}{Grossmami}
Wer kennt sie nicht? Sablé. Zartschmelzend versüssen sie jedem Guetzliliebhaber die Advendszeit.
\\
\\
\\
\begin{tabularx}{\linewidth}{r>{\bfseries\textbf}lX}
	250 g & Butter & zimmerwarm, geschmeidig rühren.\\
	150 g & Puderzucker &\\
	2 Prisen & Salz &\\
	1 Packung & Vanillezucker & Alles beigeben und rühren.\\
	500 g & Mehl & portionsweise darunterkneten, bis eine gleichmässige Masse ensteht, die sich leicht und geschmeidig bearbeiten lässt.\newline \newline
	Rollen mit 4 cm Durchmesser formen, in Klarsichtfolie einpacken und zwei Stunden im Kühlschrank aushärten lassen.\newline \newline
	Ofen auf 180 $^{\circ}$C Ober-/Unterhitze vorheizen. \newline \newline
	1 cm dicke Scheiben schneiden und auf Backblechpapier ausbreiten. Im vorgeheizten Ofen 10-12 Minuten backen lassen, bis die Ränder sich goldbraun färben.
\end{tabularx}
\\
\\
\\
\textbf{Tipps}
\begin{itemize}
	\item Sablés noch warm in Schockoladenkuvertüre tauchen und auf einem Gitter abkühlen lassen.
\end{itemize}
\newpage

%Chräbeli
\section{Chräbeli}
\shorttopPic{Bilder/chraebeli}{2.09cm}{1.09cm}{90 Minuten}{5 Personen}{Betty Bossi}
Klassisches Anisrezept nach Betty Bossi. Perfekt zu Kaffee und Tee. In den Wintermonaten schmecken sie am besten und sind somit ein definitives Must für das jährliche Guetzli-Sortiment.
\\
\\
\\
\begin{tabularx}{\linewidth}{r>{\bfseries\textbf}lX}
	4-5 & Eier & und\\
	500 g & Puderzucker & in eine Schüssel geben und sehr gut schaumig rühren.\\
	1 Prise & Salz &\\
	1 Esslöffel & Kirsch & \\
	1.5 Esslöffel & Anis & \\
	500 g & Mehl & Alles beigeben und zusammenkneten.\newline \newline
	Rollen mit 1.5 cm Durchmesser formen, in 5 cm lange Stücke schneiden und 2-3 mal leicht schräg einschneiden. Die leicht gebogenen Stücke auf ein Blech verteilen und über Nacht aushärten lassen.\newline \newline
	Ofen auf 140 $^{\circ}$C Ober-/Unterhitze vorheizen. Chräbeli bei leicht geöffnetem Ofen im unteren Drittel 20-25 Minuten backen bis die Spitzen sich leicht bräunlich verfärben.\\
	
\end{tabularx}
\\
\\
\\
\textbf{Tipps}
\begin{itemize}
	\item Wenn die Guetzlibox mit einer Frischhaltefolie abgedichtet wird, bleiben die Chräbeli länger frisch.
\end{itemize}
\newpage


% Amaretti
\section{Amaretti}
\shorttopPic{Bilder/amarettiHomeMade}{7.1cm}{11.8cm}{25 + 190 Minuten}{etwa 40 Stück}{Linda Gratwohl}
Amaretti (``leicht bitter'') werden im Piemont seit Mitte des 17. Jahrhunderts genossen. Den charakteristischen zartbitteren Geschmack erhalten die Amaretti durch die Bittermandeln.
\\
\\
\\
\begin{tabularx}{\linewidth}{r>{\bfseries\textbf}lX}
	200 g & Zucker & \\
	250 g & gemahlene Mandeln & \\
	3 Esslöffel & Kakaopulver &\\
	1 Esslöffel & Mehl & Alle Zutaten gut mischen.\\
	2 & Eier &\\
	2 ml & Bittermandelaroma & Beigeben und mischen bis eine gleichmässige Masse entsteht.\newline\newline
												Die Masse im Kühlschrank aushärten lassen. Nach 3 Stunden herausnehmen und in baumnussgrosse Kugeln formen. Diese etwas flachdrücken und grosszügig in Puderzucker wenden. Mit genügend Abstand auf ein Backpapier verteilen und weitere 10 Minuten kaltstellen. Den Ofen auf 200 $^{\circ}$C Ober-/Unterhitze vorheizen.\newline\newline
Die Amaretti 10 Minuten auf der zweituntersten Rille backen.
\end{tabularx}
\\
\\
\\
\textbf{Tipps}
\begin{itemize}
	\item Für Schokolade-Rum-Amaretti Kakaopulver durch Schokoladenpulver und Bittermandelaroma durch Rumaroma ersetzen.
	\item Für Vanille-Amaretti 180~g Zucker und 2 Esslöffel Mehl verwenden. Kakaopulver durch Puderzucker und Bittermandelaroma durch Vanillearoma ersetzen.
\end{itemize}
\newpage


%Schoggi-Chüechli
\section{Schoggi-Chüechli}
\shorttopPic{Bilder/schoggiChuechli}{4.4cm}{3.06cm}{30 + 20 Minuten}{15 Stück}{Mam}
Hier werden Kindheitserinnerungen wach. Nur schon der Geruch im Haus erinnert an Geburtstage in den 90er-Jahren. Ein herrliches Rezept für jeden Anlass.
\\
\\
\\
\begin{tabularx}{\linewidth}{r>{\bfseries\textbf}lX}
	& & Ofen auf 180 $^{\circ}$C Ober-/Unterhitze vorheizen.\\
	125 g & Margarine &\\
	125 g & Zucker &\\
	1 Teelöffel & Vanillezucker &\\
	1 Prise & Salz &\\
	3 & Eier & Alles schaumig rühren.\\
	125 g & dunkle Schokolade & im Wasserbad flüssig werden lassen und zusammen mit\\
	125 g & gemahlene Haselnüsse & und\\
	50 g & Mehl & unterrühren.\newline \newline Masse in Förmchen abfüllen und 20 Minuten backen lassen.\\
\end{tabularx}
\\
\\
\\
\textbf{Tipps}
\begin{itemize}
	\item Förmchen auf dem Blech nicht zu eng aufreihen.
	\item Das Portionieren geht besonders einfach mit einem Eisportionierer.
\end{itemize}
\newpage


%Himbeer-Muffin
\section{Himbeer-Schoggi-Muffin}
\shorttopPic{Bilder/rasberryMuffin}{0.7cm}{0.4cm}{30 + 20 Minuten}{12 Stück}{trial and error}
Dieses Rezept habe ich in Schweden einige Male ausprobiert und verfeinert. Es wurde jedes Mal zum Erfolg.
\\
\\
\\
\begin{tabularx}{\linewidth}{r>{\bfseries\textbf}lX}
	& & Ofen auf 180 $^{\circ}$C Ober-/Unterhitze vorheizen.\\
	100 g & weisse Schokolade & in Würfel zerkleinern\\
	150 g & Zucker &\\
	300 g & Mehl &\\
	1 Prise & Salz &\\
	1 Packung & Backpulver & Alles zusammen gut mischen.\\
	1 & Ei &\\
	125 ml & Planzenöl &\\
	180 ml & Milch & Zusammen zerquirlen.\newline \newline Alles zusammengeben und zu einer homogenen Masse rühren.\\
	125 g & tiefgefrorene Himbeeren & beifügen und nur grob untermischen.\newline \newline Masse in Förmchen abfüllen und 20 Minuten backen lassen.
\end{tabularx}
\\
\\
\\
\textbf{Tipps}
\begin{itemize}
	\item Förmchen auf dem Blech nicht zu eng aufreihen.
	\item Das Portionieren geht besonders einfach mit einem Eisportionierer.
\end{itemize}
\newpage


%Marmorkuchen
\section{Marmorkuchen}
\shorttopPic{Bilder/marmorcake}{1.5cm}{0.92cm}{40 Minuten}{Eine längliche Kuchenform}{Lindas Mutter}
Ein wunderbar leichtes Rezept für einen luftigen Marmorkuchen. Immer wieder ein Genuss.
\\
\\
\\
\begin{tabularx}{\linewidth}{r>{\bfseries\textbf}lX}
	& & Ofen auf 180 $^{\circ}$C Ober-/Unterhitze vorheizen.\\
	4 & Eier &\\
	200 g & Zucker &\\
	1 Prise & Salz &\\
	${}^1/_2$ Pack & Vanillezucker & alles zusammen während 10-15 Minuten schaumig rühren.\newline\\
	150 g & Butter & schmelzen.\\
	1 dl & Milch & dazugiessen und beiseite stellen.\\
	200 g & Mehl & zusammen mit\\
	10 g & Backpulver & durch ein Sieb abwechselnd mit der geschmolzenen Butter zur Eiermasse beigeben.\\
	100 g & dunkle Schokolade & in Wasserbad stellen und schmelzen lassen.\newline \newline
		Geschmolzene Schokolade unter die eine Hälfte der Masse mischen.\newline \newline		Auf der untersten Rille 40-50 Minuten backen lassen.
\end{tabularx}
\\
\\
\\
\textbf{Tipps}
\begin{itemize}
	\item Besonders gut wird der Kuchen mit Cr\'emant-Schokolade.
	\item Als Form eignet sich besonders gut eine längliche Kuchenform.
\end{itemize}
\newpage


%Rüeblichueche
\section{Rüeblichueche}
\shorttopPic{Bilder/rueblichueche}{2cm}{9.2cm}{45 + 50 + 10 Minuten}{Eine mittelgrosse Springform}{Betty Bossi}
Ein wunderbarer Klassiker, welcher keine weiteren Einführung braucht.
\\
\\
\\
\begin{tabularx}{\linewidth}{r>{\bfseries\textbf}lX}
	& & Ofen auf 180 $^{\circ}$C Ober-/Unterhitze vorheizen.\\
	5 & Eigelb & und\\
	250 g & Zucker & schaumig rühren.\\
	250 g & Rüebli & fein raffeln\\
	120 g & geriebene Haselnüsse &  \\
	120 g & geriebene Mandeln & \\
	1 & Zitrone & Schale abreiben, Zitrone auspressen und zusammen mit allen obenstehenden Zutaten gut mischen.\\
	80 g & Mehl & und\\
	1 Esslöffel & Backpulver & dazusieben.\\
	5 & Eiweisse & steif schlagen und zusammen mit\\
	1 Prise & Salz & vorsichtig darunterziehen.\newline\newline
	Eine runde Kuchenform mit Backpapier auskleiden und Masse einfüllen. 45-50 Minuten auf der zweituntersten Rille backen.\\\\

	250 g & Puderzucker & und\\
	3 Esslöffel & Zitronensaft & für die Glasur gut mischen, bis keine Klümpchen mehr auszumachen sind. Auf dem ausgekühlen Kuchen schön verteilen und nach Belieben verzieren.
\end{tabularx}
\\
\\
\\
\textbf{Tipps}
\begin{itemize}
	\item Wer es wirklich süss mag, kann noch 50\,g Schokostücke hinzufügen. Die Kuchenmasse ist jedoch so luftig, dass alle Schokostücke sofort absinken.
%	\item Zur Glasur ein wenig Lebensmittelfarbe beigeben.
%	\item Marzipanrüebli sind selbstverständlich ein Klassiker.
%	\item Mit gehackten Haselnüssen oder Mandeln den äusseren Rand verzieren.
\end{itemize}
\newpage


%Schoggichueche
\section{Schoggichueche}
\shorttopPic{Bilder/sabinaKuchen}{8cm}{3.8cm}{30 + 40 Minuten}{Eine mittelgrosse Springform}{Sabina Rüegg}
Ein wunderbares Rezept für einen zarten und aromatischen Schockoladenkuchen.
\\
\\
\\
\begin{tabularx}{\linewidth}{r>{\bfseries\textbf}lX}
	& & Ofen auf 180 $^{\circ}$C Ober-/Unterhitze vorheizen.\\
	200 g & Schokolade &\\
	100 g & Butter &\\
	3 Esslöffel & Milch & Alles in Pfanne bei niedriger Temeratur zergehen lassen.\\
	200 g & Puderzucker &\\
	5 & Eigelb & zusammen mit dem Puderzucker schaumig rühren.\\
	100 g & Mehl & zugeben und mit der Schokoladen-Masse vermengen.\\
	5 & Eiweisse & steif schlagen und vorsichtig darunterziehen.\newline\newline
	Eine runde Kuchenform mit Backpapier auskleiden und mit der Masse füllen.\newline\newline30-40 Minuten auf der untersten Rille backen.\\
\end{tabularx}
\\
\\
\\
\textbf{Tipps}
\begin{itemize}
	\item Besonders gut wird der Kuchen mit Cr\'emant-Schokolade.
\end{itemize}
\newpage


%Biskuitteig
\section{Biskuitteig}
\shorttopPic{Bilder/bisquitTeig}{4cm}{1.22cm}{25 + 15 Minuten}{Springform mit 24-26 cm \(\varnothing\)}{Margherit Brusco}
Die Grundlage für jede Erdbeer-Roulade, Schwarzwäldertorte, Einfache Topfen oder wie sie alle heissen. Und wie so oft verlasse ich mich hier auf ein wunderbares Italienisches Rezept. Einfach zum Verlieben!
\\
\\
\\
\begin{tabularx}{\linewidth}{r>{\bfseries\textbf}lX}
	& & Ofen auf 180 $^{\circ}$C Ober-/Unterhitze vorheizen.\\
	3 & Eiweisse & sehr steif schlagen.\\
	1 Prise & Salz &\\
	90 g & Zucker & darunterschlagen, bis die Masse glänzt.\\
	3 & Eigelb & mitrühren.\\
	90 g & Mehl & portionsweise darunterziehen. Hier ist höchste Vorsicht geboten.\newline \newline
	Boden der Springform mit Backpapier belegen. Rand muss nicht ausgelegt oder gefettet werden. Nach dem Backen kann das Biskuit ganz leicht mit dem Spachtel gelöst werden.\newline \newline
	Masse in die Springform geben und in der Mitte des Ofens während 15 Minuten backen lassen.
\end{tabularx}
\\
\\
\\
%\textbf{Tipps}
%\begin{itemize}
%\end{itemize}

\newpage

%Apfelwähe
\section{Apfelwähe}
\shorttopPic{Bilder/apfelwaehe}{4cm}{9.2cm}{15 + 30 Minuten}{Springform mit 24-26 cm \(\varnothing\)}{trial and error}
Die wohl klassischste Variante der Wähe. Locker angelehnt an das Grundrezept von Betty Bossi.
\\
\\
\\
\begin{tabularx}{\linewidth}{r>{\bfseries\textbf}lX}
	& & Ofen auf 200 $^{\circ}$C Ober-/Unterhitze vorheizen.\\
	1 grosses & Ei &\\
	2 Esslöffel & Zucker &\\
	${}^1/_2$ Prise & Salz &\\
	1.5 dl & Milch &\\
	1 Teelöffel & Zimt & Alles mit einem Schwingbesen zu einem hellen Guss schlagen.\newline \\
	1 & Blätterteig & in eine runde Springform auslegen und dicht ausstechen.\\
	& gemahlene Mandeln & deckend auf den Boden streuen.\newline \\
	2.5 & Äpfel & schälen und in 12 gleich grosse Stücke schneiden.\newline \newline
		Die Apfelschnitze gleichmässig auf den Boden des Blätterteigs verteilen. Den Guss gleichmässig über die Apfelschnitze verteilen.\newline \newline
		In der Mitte des Ofens etwa 30 Minuten backen lassen.\\
	
\end{tabularx}
\\
\\
\\
\textbf{Tipps}
\begin{itemize}
	\item Entweder 1 grosses oder 2 kleine Eier nehmen.
\end{itemize}
\newpage

%Apfelwähe
\section{Lebkuchen}
\shorttopPic{Bilder/lebkuchen}{9.6cm}{9.3cm}{20 + 30 Minuten}{1 Blech}{Mam}
Dieses traditionelle Weihnachtsgebäck darf auf keinem Adventsfenster-Event oder Weihnachtsmarkt fehlen. Ein herrlicher Klassiker, der es versteht, die vielen Düfte und Geschmäcker der Weihnachtszeit in einem einzigen Biss zu vereinen.\\
\\
\\
\begin{tabularx}{\linewidth}{r>{\bfseries\textbf}lX}
	& & Ofen auf 180 $^{\circ}$C Ober-/Unterhitze vorheizen.\\
	${}^1/_2$ l & Milch &\\
	${}^1/_2$ kg & Zucker &\\
	2-3 Esslöffel & Kakao &\\
	1 Beutel & Lebkuchengewürz &\\
	4 Esslöffel & Öl & Alles mit einem Schwingbesen zu einem hellen Guss schlagen.\newline \\
	${}^1/_2$ kg & Mehl & in eine runde Springform auslegen und dicht ausstechen.\\
	1 Päckli & Backpulver & 30 Minuten 180 $^{\circ}$C Ober-/Unterhitze vorheizen.
\end{tabularx}
\\
\\
\\
\textbf{Tipps}
\begin{itemize}
	\item Entweder 1 grosses oder 2 kleine Eier nehmen.
\end{itemize}
\newpage



%
%
%
%Besonderheiten
%
%
%
\fancyhead[R]{\small\scshape Besonderheiten}

\part{Besonderheiten}

%Glühwein
\section{Glühwein}
\shorttopPic{Bilder/gluehwein}{6cm}{3.87cm}{30 Minuten}{1.5 Liter}{Mam}
Ein echtes Wintergetränk.
\\
\\
\\
\begin{tabularx}{\linewidth}{r>{\bfseries\textbf}lX}
	1 l & schwerer Rotwein &\\
	120 g & Zucker &\\
	2-3  & Nelken &\\
	2 & Zimtstangen &\\
	${}^1/_2$ & Orange &\\
	${}^1/_2$ & Zitrone & Die Orange und die Zitrone auspressen. Den Saft zusammen mit den anderen Zutaten in einen Topf geben und bis kurz vors Kochen bringen. Sieben und heiss servieren.
\end{tabularx}
\\
\\
\\
\textbf{Tipps}
\begin{itemize}
	\item Falls der Glühwein ohne seine erheiternde Wirkung genossen werden will, empfiehlt es sich, Traubensaft statt Rotwein zu verwenden. Die Zuckermenge sollte hierbei jedoch reduziert werden.
\end{itemize}
\newpage


\end{document}
